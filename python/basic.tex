\documentclass{article}%
% Default fixed font does not support bold face
\DeclareFixedFont{\ttb}{T1}{txtt}{bx}{n}{12} % for bold
\DeclareFixedFont{\ttm}{T1}{txtt}{m}{n}{12}  % for normal

% Custom colors
\usepackage{color}
\definecolor{deepblue}{rgb}{0,0,0.5}
\definecolor{deepred}{rgb}{0.6,0,0}
\definecolor{deepgreen}{rgb}{0,0.5,0}
\definecolor{deeporange}{rgb}{1, 0.549, 0}
\definecolor{deeppurple}{rgb}{0.471, 0.314, 0.663}
\definecolor{bgcolor}{rgb}{0.95,0.95,0.92}
\usepackage{listings}

% Python style for highlighting
\newcommand\pythonstyle{\lstset{
backgroundcolor=\color{bgcolor},
language=Python,
basicstyle=\ttm,
otherkeywords={self},             % Add keywords here
keywordstyle=\ttb\color{deeppurple},
emph={[1] for, in, if, else, },          % Custom highlighting
emph={[2]exampleFunction},          % Custom highlighting
emphstyle=\ttb\color{deeporange},    % Custom highlighting style
emphstyle=[2]\ttb\color{deepblue},    % Custom highlighting style
stringstyle=\color{deepgreen},
frame=tb,                         % Any extra options here
showstringspaces=false,            % 
breakatwhitespace=false,
breaklines=true,
captionpos=b,
keepspaces=true,
numbers=left,
showtabs=false,
tabsize=4
}}


% Python environment
\lstnewenvironment{python}[1][]
{
\pythonstyle
\lstset{#1}
}
{}

% Python for external files
\newcommand\pythonexternal[2][]{{
\pythonstyle
\lstinputlisting[#1]{#2}}}

% Python for inline
\newcommand\pythoninline[1]{{\pythonstyle\lstinline!#1!}}

\usepackage[T1]{fontenc}%
\usepackage[utf8]{inputenc}%
\usepackage{lmodern}%
\usepackage{textcomp}%
\usepackage{lastpage}%
%
%
\title{Utbildningsplan Python}
\author{Mats Larsson}
\begin{document}%

\maketitle
\normalsize%
\section{Lektion 1}%
\paragraph{Presentera mig själv}
\paragraph{Gå igenom utbildningsplanen de kommande veckorna}
\subsection{Python}
\begin{itemize}
\item Världens populäraste programmeringsspråk
\item Lätt att lära sig
\item Python 2 och python 3
\end{itemize}

TODO: slides
\subsection{IDLE}
IDLE är pythons egna editor. En skribent använder Word och en programmerare använder IDLE, eller
någon annan av de hundratals olika editorer. IDLE består av två olika fönster: skalet och filen.
I filen skriver vi koden och i skalet input och outputs. 


\begin{python}[language=Python, caption=Hello World]
print("Hello World!")

\end{python}


\subsection{Zifro}
www.zifro.se/hello/
\subsubsection{Styr Podden}

\begin{python}[language=Python, caption=for-loop]
for i in range(5):
    print(i)
 
\end{python}

\subsubsection{Räkna varor}

\begin{python}[language=Python, caption=Variabler] 
x = 3+4 
y = 6/2
z = 2*3
print(x+y-z)
\end{python}

\Large I python skrivs multiplikationstecknet \textbf{$\cdot$} som * \hfill
\normalsize

\begin{python}[language=Python, caption=Ökad mängd] 
exempel = 0

for i in range(5):
    exempel = exempel + i

print(exempel)
\end{python}

\subsubsection{Sortera varorna}

\begin{python}[language=Python, caption=if-satser]
if(3 == 3):
    print("Foo")
else:
    print("Bar")
\end{python}

\subsection{Slack}
TODO: workspace


\section{Lektion 2}
\subsection{Repitition}
\begin{itemize}
\item variabler
\item for-loopar
\item if-satser
\end{itemize}

\subsection{Zifro}
\subsubsection{Chatbotten}
\paragraph{Datatyper}
\begin{itemize}
\item Strängar
\item Integraler
\item Double/Float
\item Boolean
\end{itemize}
\subsubsection{Miniräknare}

\begin{python}[language=Python, caption=while-loop]
go = True

while(go):
    print(foo) 
\end{python}


\section{Lektion 3}
\subsection{Repitition}

\subsection{Zifro}
\subsubsection{Talföljder}
\subsubsection{Delbarhet}



\section{Lektion 4}

\subsection{Repitition}
\subsection{Funktioner}
\begin{python}[language=Python, caption=Funktioner]
def exampleFunction(x, y):
 
    return x+y

x = 3
y = 4
z = exampleFunction(x, y)

print(z)
\end{python}
\section{Lektion 5}
\subsection{Repitition}
\section{Lektion 6}
\subsection{Repitition}
\section{Lektion 7}
\subsection{Repitition}
\section{Lektion 8}
\subsection{Repitition}
\section{Lektion 9}
\section{Lektion 10}
\section{Extra uppgifter}
%
\end{document}
